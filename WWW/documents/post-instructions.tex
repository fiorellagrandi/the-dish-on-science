%%%%%%%%%%%%%%%%%%%%%%%%%%%%%%%%%%%%%%%%%%%%%%%%%%%%%%%%%%%%%%%%%%%%%%
% Instructions for Formatting and Posting Articles to
% thedishonscience.com
%
%%%%%%%%%%%%%%%%%%%%%%%%%%%%%%%%%%%%%%%%%%%%%%%%%%%%%%%%%%%%%%%%%%%%%%
% Edit the title below to update the display in My Documents
\title{How to Post}
%
%%% Preamble
\documentclass[paper=a4, fontsize=11pt]{scrartcl}
\usepackage[T1]{fontenc}
\usepackage{fourier}

\usepackage[english]{babel}                                                            % English language/hyphenation
\usepackage[protrusion=true,expansion=true]{microtype}
\usepackage{amsmath,amsfonts,amsthm} % Math packages
\usepackage[pdftex]{graphicx}
\usepackage{hyperref}
\usepackage{forest}
\usepackage{float}


%%% Custom sectioning
\usepackage{sectsty}
\allsectionsfont{\centering \normalfont\scshape}


%%% Custom headers/footers (fancyhdr package)
\usepackage{fancyhdr}
\pagestyle{fancyplain}
\fancyhead{}                                            % No page header
\fancyfoot[L]{}                                            % Empty
\fancyfoot[C]{}                                            % Empty
\fancyfoot[R]{\thepage}                                    % Pagenumbering
\renewcommand{\headrulewidth}{0pt}            % Remove header underlines
\renewcommand{\footrulewidth}{0pt}                % Remove footer underlines
\setlength{\headheight}{13.6pt}


%%% Equation and float numbering
\numberwithin{equation}{section}        % Equationnumbering: section.eq#
\numberwithin{figure}{section}            % Figurenumbering: section.fig#
\numberwithin{table}{section}                % Tablenumbering: section.tab#
\newcommand{\dishurlplain}[1]{http://thedishonscience.stanford.edu/#1}
\newcommand{\dishurl}[1]{\url{\dishurlplain{#1}}}


%%% Maketitle metadata
\newcommand{\horrule}[1]{\rule{\linewidth}{#1}}     % Horizontal rule

\title{\
        %\vspace{-1in}
        \usefont{OT1}{bch}{b}{n}
        % \normalfont{}\normalsize \textsc{School of random department names} \\ [25pt]
        \horrule{0.5pt} \\[0.4cm]
        \huge How to Post on \\ \dishurl{} \\
        \horrule{2pt} \\[0.5cm]
}
\author{\
        \normalfont{}                     \normalsize
        Bruno Beltran\\[-3pt]             \normalsize
        \today
}
\date{}


%%% Begin document
\begin{document}
\maketitle

\section{Summary}
Before submitting an article to the Editor, please follow the following steps:
\begin{enumerate}
    \item Fill out the
        \href{\dishurlplain{documents/post\_info.xlsx}}{post information
            Excel sheet}. See Section~\ref{sec:excel} for more information.
    \item Copy and paste the article text into StackEdit
    (\url{https://stackedit.io/editor}) and make sure that it looks
    correct when interpreted as Markdown, see Section~\ref{sec:markdown} for more
    information.
        \begin{itemize}
            \item Make sure references and external links are hyperlinked
                properly.
            \item Make sure that your figures appear where you want them to be.
            \item Make sure you follow all conventions in
                Section~\ref{sec:conventions}.
        \end{itemize}
    \item Make a main folder for your blog post. Give it the same name as what
        you put in the ``URL'' field of the Excel document.

    \item Save your post from StackEdit as \texttt{post.md} in this folder.
        \begin{itemize}
            \item To export from StackEdit, click on the hashtag in the top left
                corner, select ``Export to Disk'' and click on ``As Markdown''.
        \end{itemize}
    \item Save all your images in a subfolder called \texttt{images}. Make sure
        that the image name matches the link in the \texttt{post.md} Markdown
        file.
    \item Put your post information Excel document into the folder, zip it, and
        email to the editor for review.
\end{enumerate}

\section{What Can Go in My Article?}
The Dish on Science is a Stanford graduate student blogging group.
As a Stanford graduate student group, we must adhere to Stanford's
\href{https://adminguide.stanford.edu/chapter-6/subchapter-2/policy-6-2-1}{``Computer
and Network Usage Policy''}. In short, this means that in your official writing
for the Dish:\@
\begin{enumerate}
    \item DO NOT expose anyone's online identity.
    \item DO NOT use The Dish to advocate for yourself or a business you know,
        even in passing.
    \item DO NOT explicitly advocate for any political group.
    \item DO NOT violate copyright law (see
        Section~\ref{sec:copyright} below for details).
\end{enumerate}

As long as you keep to the simplified guidelines above and maintain a
professional tone, we should not have any problems. If in doubt, however,
consult the full university policy at
\url{https://adminguide.stanford.edu/chapter-6/subchapter-2/policy-6-2-1}.

\section{Copyright Law}\label{sec:copyright}

This is fairly easy to summarize. If you didn't make it, and it is not both from
a credible source and explicitly marked as free for reuse, then you have to ask
the person that owns it for permission to use it. Don't assume that something
marked free for reuse at \texttt{sketchy.illegalwebsite.wut} is actually
legitimately free for reuse.  Besides saying explicitly somewhere that you're
free to reproduce it, the most common way a content creator will give you legal
rights to reuse something is to mark it as being under a Creative Commons
license or as being ``public domain''.

In particular, this means that if you want to reproduce a figure from a journal,
you invariably \textbf{must} ask permission from the copyright holder. This is
usually the journal, but you can find out for sure by searching either the PDF
or the webpage of the article for a Copyright notice for the article and seeing
who the copyright is assigned to. Do not mistake the article being available
from a separate site as meaning that it is okay to steal pictures from it. Many
good, well-meaning blogs have been sued for exactly this practice.

If you do procure permission from the owner of content that you want to use in
your post, say a peer or PI that you are acquaintances with, please include a
copy of the permission in writing in your submission folder when you send your
article to the editor.

Finally, if you \textit{absolutely must} get permission from an academic journal
to use a figure from a paper that you can't just reproduce, please contact the
head of The Dish or the editor so that she or he may procure permission in the
name of the organization.

As an important aside, realize that this document is not legally binding in any
way and its author is not a lawyer, so please use your own best judgement as
necessary to obey all relevant laws and regulations.

\section{How to Format a Post}
The Dish's website is maintained by the Website Administrator, and the Editor in
Chief is in charge of managing timely submission of articles to the site, per
their duties as outlined in
\href{\dishurlplain{documents/dish-constitution.pdf}}{the constitution}. A
large part of the uploading is automated to both ensure consistency of design
across the site and to minimize the extra work that has to be done by the
Editor.  As such, we ask that all posts follow the following format
\textit{exactly}.

A minimal, correct example article as the Editor will expect to recieve it can
be found at \dishurl{documents/minimal-example}. A analagous post that
leverages all optional features of the website that are available without
special requests can be found at \dishurl{documents/maximal-example}. In
places where this document is ambiguous, these examples should serve as an
official reference.

For an article with desired URL
\dishurlplain{posts/post-name-url}, the folder structure in
Figure~\ref{fig:folder-structure} is required. The
editor will expect to recieve a single file, \texttt{post-url-name.zip}, with
the entire contents of \texttt{post-url-name} folder.
\begin{figure}[h]
\begin{forest}
  for tree={%
    font=\ttfamily,
    grow'=0,
    child anchor=west,
    parent anchor=south,
    anchor=west,
    calign=first,
    edge path={%
      \noexpand\path{} [draw, \forestoption{edge}]
      (!u.south west) + (7.5pt,0) |- node[fill,inner sep=1.25pt] {} (.child anchor)\forestoption{edge label};
    },
    before typesetting nodes={%
      if n=1
        {insert before={[,phantom]}}
        {}
    },
    fit=band,
    before computing xy={l=15pt},
  }
[post-url-name
  [images
    [image1.png]
    [image2.svg]
    [\ldots{}]
  ]
  [post.md]
  [post\_info.xlsx]
]
\end{forest}
\caption{Post folder structure guidelines.}\label{fig:folder-structure}
\end{figure}

The \texttt{post\_info.xlsx} and \texttt{post.md} files must have
\textbf{exactly those names}. The image files can be called anything, as long as
they're correctly linked to in the article, but the
folder containing them must be called \texttt{images}. See Section~\ref{sec:image-links}
for how to correctly link to your images.

\emph{For advanced users only:} for full control of the article using custom
HTML, CSS, or JavaScript, simply include a your own \texttt{post.html} in
the top level \texttt{post-url-name} directory. This will
prevent the server from attempting to compile one for you from \texttt{post.md}.
You may assume that the webserver has read-only access to arbitrary
subdirectories of your post directory.

\subsection{How to Fill in post\_info.xlsx}\label{sec:excel}
Please see the examples mentioned above for how to fill in the file. If any
field does not apply to your article, please \textit{leave it blank}. \textbf{Do
not} put
extraneous ``N/A'' marks in cells that you do not use.

All image links should be made relative to the article
directory. More explicitly, please use
``./images/\textbf{image-file-name}.\textbf{png}''
filling in the correct filename and extension (\texttt{.svg}, \texttt{.jpg},
etc.) each time a link to an image file is required.

\noindent{}The required fields are:\@
\begin{itemize}
    \item \emph{Post Title} : Must be below 200 characters.
    \item \emph{Post URL} : Must be below 200 characters.
        Must be all lowercase letters and words should be separated by single
        hyphens (i.e.\ the ``-'' character, ASCII character \texttt{0x2D}). Do
        not use spaces in the post URL.
    \item \emph{5x2 (WxH) Image File Name} : A link to the main image that will be
        displayed alongside links to the article and at the top of the article.
        This image will look best if made with an aspect ratio of 5:2 width to
        height, at least 300px of height, and at least 800px of width. A stock
        example can be found at \dishurl{images/placeholder-5to2.jpg}.
    \item \emph{Post ``Blurb''} : A short, sub-sentence-long description of what
        the article is about. If you want your article to have a subtitle, this
        is the most appropriate place. This can be at most 100 characters
        including spaces.
    \item \emph{Post Description} : A short paragraph that will be displayed
        beneath your article's image when listing articles. This should be your
        call to action, and after reading this, a visitor to the site should
        want to click on your article to read more. The text ``(Continue
        reading\ldots{})'' will be automatically included after this description, so do
        not include it yourself. This can be at most 500 characters including
        spaces. Shorter is usually better.
    \item \emph{Team Name}: The official url of the team or teams that have edited the
        post. For example, this might be ``biochemistry-and-bioinformatics'' or
        ``general-biology'', but \textbf{not} ``Biochemistry / Bioinformatics'' or
        ``Biology (General)''. You can find your teams official URL by navigating
        to your team's page on \dishurl{} and copying the part of the URL
        that comes after \texttt{/topics/}.
\end{itemize}

\noindent{}The following (optional) fields are also allowed.
\begin{itemize}
    \item \emph{Author information}:
        \begin{itemize}
            \item \emph{Author name}: Please use your full legal name to make this consistent. If
                you submit one article as ``Bob Caldwell'' and another as ``Bobby
                Caldwell'', there will be no way for the web server to know that
                these should be assigned to the same person. You should think of
                this as your private username for The Dish.
                If this field is omitted, the group name will be used as a stand in for
                the author name.
            \item \emph{Author nickname} : One per author. This will be the name that is
                actually displayed on the site next to your (optional) picture. It can
                be different for each article you write if you wish. You are
                free to use spellings that require non-standard unicode
                characters (e.g. Japanese, Hebrew, or Greek script is okay).
            \item \emph{Author headshot} : This will be a link to the picture
                that will display next to your name.  You can provide an image
                with your article and link to that file name inside the
                ``\texttt{./images}'' folder. You have several options if you
                don't want to actually put a picture of yourself. You can use
                any of
                \begin{itemize}
                    \item \texttt{./images/cow.png}
                    \item \texttt{./images/dinosaur.png}
                    \item \texttt{./images/elephant.png}
                    \item \texttt{./images/hedgehog.png}
                    \item \texttt{./images/monkey.png}
                    \item \texttt{./images/panda.png}
                    \item \texttt{./images/paperplane.png}
                    \item \texttt{./images/penguin.png}
                    \item \texttt{./images/turkey.png}
                \end{itemize}
                or leave this field blank for a random picture of a
                cute animal.
        \end{itemize}
        Note that if there are multiple authors, you can simply use a new Excel
        column for each author.
    \item \emph{Illustrator information} : Same as
        the ``author'' fields, but for the illustrator(s).
    \item \emph{2x1 (WxH) Image File Name} : A link to a cropped version of the post image that
        will be used when the post is displayed on the left sidebar. This image
        must have an aspect ratio 2:1 of width:height or it will be ignored by
        the server. If it is not less than
        200px in height, it will be downsampled. If this is not included, it
        will automagically be generated from the ``5x2 (WxH) Image File Name''.
        It will likely look bad. The website author takes no responsibility for
        how ridiculous the cropped image looks if you don't crop it yourself. A stock
        example can be found at
        \dishurl{images/placeholder-2to1.jpg}.
    \item \emph{1x1 (WxH) Image File Name (Thumbnail)} : A link to a cropped version of the post image
        that can be used as a thumbnail for the post. This image must have an
        aspect ratio of 1:1 or it will be ignored. If it is larger than 50px in
        height, it will be downsampled. A stock
        example can be found at
        \dishurl{images/placeholder-1to1.jpg}.

\end{itemize}

\subsection{How to Make \texttt{\textbf{post.md}}}\label{sec:markdown}

\subsubsection{Basic Markdown}

Each group is free to (and should) use whatever format (e.g.\ Word, Google Docs, plain text,
LaTeX+git) makes it easiest for your groups to get the articles written and
reviewed. When it comes time for you to submit, however, the process of making
the article look like you want it to on the webpage will require that you use a web-friendly
format. In order to minimize the amount of work this will take both for the
authors and the editor, we will use ``Markdown'' syntax to specify the article's
formatting.

The best way to explain how it works is to simply point you to a cheat
sheet that shows you how to do everything you could possibly want,
from italics and bolding to tables, links and images:

\url{https://github.com/adam-p/markdown-here/wiki/Markdown-Cheatsheet}

Here's a site that lets you type in Markdown and shows you what it would
look like on the web, with some built in examples.

\url{https://stackedit.io/editor}

In practice, the only thing to keep in mind is that you'll need to put
two newlines (i.e.\ hit enter twice) every time you want to start a new
paragraph, and everything else can be looked up easily on the
\href{https://github.com/adam-p/markdown-here/wiki/Markdown-Cheatsheet}{cheatsheet}.

Various people have found
that a relatively efficient way to get the article formatted is to just copy/paste
from Word into \href{https://stackedit.io/editor}{StackEdit} and fix the few problems that appear. More advanced users
might want to try using an automatic conversion tool like
\href{http://pandoc.org/demos.html}{Pandoc} or the
\href{http://www.writage.com/}{``Writage''} plugin. Regardless of how you choose
to write the Markdown text, please always check the formatting on StackEdit.

\textit{Important tip:} \textbf{DO NOT use Word. DO NOT use TextEdit.}
Word and TextEdit will corrupt your markdown file, just like they corrupt FASTA
files if you try to edit then save them. For any small changes after you
download the Markdown file from StackEdit, please use one of the following
programs:
\begin{itemize}
    \item On Windows, I
recommend downloading \href{https://notepad-plus-plus.org/}{Notepad++},
or just using Notepad if you don't feel like downloading a new program.
    \item On Mac, I recommend just downloading
\href{https://itunes.apple.com/us/app/textwrangler/id404010395?mt=12}{TextWrangler}.
    \item On Linux, any reasonalbe distro's default editor should work. I prefer
Vim.
\end{itemize}

\subsubsection{Images}\label{sec:image-links}
In order to get an image in your article, simply include it in the
``\texttt{images}'' subfolder of your post as demonstrated in
Figure~\ref{fig:folder-structure}. Then, the usual syntax for including a
picture in a Markdown document should work using the relative file path
(i.e.\ starting with ``\texttt{./images/}'').

For example, to create a link to ``\texttt{image1.png}'' in the example in
Figure~\ref{fig:folder-structure}, one would, inside of \texttt{post.md}, use
the syntax

\begin{verbatim}![alt text](./images/image1.png "hover text")\end{verbatim}

The text ``\texttt{alt text}'' will then appear as a stand-in if the image fails to
load or loads too slowly, and the text ``\texttt{hover text}'' will appear if the
reader hovers their mouse over the image.

When composing the article, it might be helpful to use an external tool like
\href{https://stackedit.io/editor}{StackEdit} to view your article as you type.
If you want to also see your images to get a rough idea of how they will appear
in the article, you have to provide a valid link to the file such that the
program that you're using can find the image. For example, when using
\href{https://stackedit.io/editor}{StackEdit}, the easiest solution is to upload
the images to an online file sharing program like Dropbox and use the dropbox
link instead of \texttt{./images/image1.png} in the example above while you're
writing the article. Just make sure that the version of the Markdown file that you
submit to the editor has the links formatted as in the example above, and not
dropbox links.

\textit{Important:} \textbf{Do not link to external sites. All images used in the article must be
included in the images folder.}

Finally, remember that on the web, portable formats are king. Vectorized graphics are
especially nice to have. In rough order of decreasing preference, please use one
SVG, PNG, or JPG/JPEG formatted files. If you have a graphic in another format,
for example from illustrator/inkscape, photoshop/gimp, or powerpoint (shame on
you!) then please convert it to one of the three above formats.

\emph{For advanced users only:} Please
export your graphic in the format that optimizes file size while preserving image quality
when viewed across 6 inches of screen (width). Use any web-friendly format.

\subsubsection{Conventions}\label{sec:conventions}
\begin{enumerate}
    \item Include neither the title of the article nor the authors' nor
        illustrators' names in the \texttt{post.md} file. They will be included
        authomatically by the website.
    \item All section titles should use ``level 3'' headers, i.e.\ lines with section
        titles should start with three ``\texttt{\#}'' characters in the
        \texttt{post.md} file. For each level of sub-section, add another
        ``\texttt{\#}'' character. See the
        \href{https://github.com/adam-p/markdown-here/wiki/Markdown-Cheatsheet}{cheatsheet}
        under ``headers'' for more details.
    \item References should be in Science format, and each reference in the
        article to a citation should link to that citation the references
        section. Each citation in the references section should in turn be a
        link to the article's page if available.
    \item Figures should be included by simply including the image, leaving
        blank link, and bolding the text of the caption.
    \item For an example of how to include footnotes, see Rohan's article at
        \dishurl{posts/anti-vaccine-sentiment-disease-dynamics/} and the
        associated markdown file at
        \dishurl{documents/existing-posts/anti-vaccine-sentiment-disease-dynamics/post.md}.

\end{enumerate}

%%% End document
\end{document}
