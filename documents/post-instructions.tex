%%%%%%%%%%%%%%%%%%%%%%%%%%%%%%%%%%%%%%%%%%%%%%%%%%%%%%%%%%%%%%%%%%%%%%
% Instructions for Formatting and Posting Articles to
% thedishonscience.com
%
%%%%%%%%%%%%%%%%%%%%%%%%%%%%%%%%%%%%%%%%%%%%%%%%%%%%%%%%%%%%%%%%%%%%%%
% Edit the title below to update the display in My Documents
\title{How to Post}
%
%%% Preamble
\documentclass[paper=a4, fontsize=11pt]{scrartcl}
\usepackage[T1]{fontenc}
\usepackage{fourier}

\usepackage[english]{babel}                                                            % English language/hyphenation
\usepackage[protrusion=true,expansion=true]{microtype}
\usepackage{amsmath,amsfonts,amsthm} % Math packages
\usepackage[pdftex]{graphicx}
\usepackage{hyperref}
\usepackage{forest}


%%% Custom sectioning
\usepackage{sectsty}
\allsectionsfont{\centering \normalfont\scshape}


%%% Custom headers/footers (fancyhdr package)
\usepackage{fancyhdr}
\pagestyle{fancyplain}
\fancyhead{}                                            % No page header
\fancyfoot[L]{}                                            % Empty
\fancyfoot[C]{}                                            % Empty
\fancyfoot[R]{\thepage}                                    % Pagenumbering
\renewcommand{\headrulewidth}{0pt}            % Remove header underlines
\renewcommand{\footrulewidth}{0pt}                % Remove footer underlines
\setlength{\headheight}{13.6pt}


%%% Equation and float numbering
\numberwithin{equation}{section}        % Equationnumbering: section.eq#
\numberwithin{figure}{section}            % Figurenumbering: section.fig#
\numberwithin{table}{section}                % Tablenumbering: section.tab#


%%% Maketitle metadata
\newcommand{\horrule}[1]{\rule{\linewidth}{#1}}     % Horizontal rule

\title{\
        %\vspace{-1in}
        \usefont{OT1}{bch}{b}{n}
        % \normalfont{}\normalsize \textsc{School of random department names} \\ [25pt]
        \horrule{0.5pt} \\[0.4cm]
        \huge How to Post on \\ \texttt{thedishonscience.org} \\
        \horrule{2pt} \\[0.5cm]
}
\author{\
        \normalfont{}                     \normalsize
        Bruno Beltran\\[-3pt]             \normalsize
        \today
}
\date{}


%%% Begin document
\begin{document}
\maketitle
\section{What Can Go in My Article?}
The Dish on Science is a Stanford graduate student blogging group.
As a Stanford graduate student group, we must adhere to Stanford's
\href{https://adminguide.stanford.edu/chapter-6/subchapter-2/policy-6-2-1}{``Computer
and Network Usage Policy''}. In short, this means that in your official writing
for the Dish:\@
\begin{enumerate}
    \item DO NOT expose anyone's online identity.
    \item DO NOT use The Dish to advocate for yourself or a business you know,
        even in passing.
    \item DO NOT explicitly advocate for any political group.
    \item DO NOT violate copyright law (see
        \hyperref[how-to-not-violate-copyright]{below} for details).
\end{enumerate}

As long as you keep to the simplified guidelines above and maintain a
professional tone, we should not have any problems. If in doubt, however,
consult the full university policy linked above.

\section{How to Format a Post}
The Dish's website is maintained by the editor as part of his or her duties. A
large part of this is automated to ensure consistency of design across the site.
As such, we ask that all posts follow the following format \textit{exactly}.

For an article with desired URL
\texttt{http://www.dishonscience.org/posts/post-name-url}, the following folder structure is required. File names \textbf{should be exactly as indicated}:


\begin{forest}
  for tree={
    font=\ttfamily,
    grow'=0,
    child anchor=west,
    parent anchor=south,
    anchor=west,
    calign=first,
    edge path={
      \noexpand\path [draw, \forestoption{edge}]
      (!u.south west) +(7.5pt,0) |- node[fill,inner sep=1.25pt] {} (.child anchor)\forestoption{edge label};
    },
    before typesetting nodes={
      if n=1
        {insert before={[,phantom]}}
        {}
    },
    fit=band,
    before computing xy={l=15pt},
  }
[post-url-name
  [images
    [image1.png]
    [image2.svg]
    [\ldots{}]
  ]
  [post.md]
  [post\_info.json ]
]
\end{forest}

\subsection{How to Make \texttt{\textbf{post\_info.json}}}
A full, working \texttt{post\_info.json} file can be found at
nc\texttt{http://www.dishonscience.org/documents/example\_post\_info.json}.
The required fields are:\@
\begin{enumerate}
    \item Post Title
    \item Main Post Image
    \item Post "Blurb"
    \item Post Description
\end{enumerate}

The following (optional) fields are also allowed.


\subsection{How to Make \texttt{\textbf{post.md}}}
Please see the following email exchange for everything you need to know.

\begin{verbatim}
Hi Kelly,

Simply export it to a picture and include the picture in the article.
Remember that for pretty pictures on the web high quality PNG or SVG
should be preferred.

Cheers,

Bruno


On Fri, Jan 22, 2016 at 3:46 PM, Kelly Jane McGill <kmcgill@stanford.edu> wrote:
> Hello,
> Is it possible to include a pictorial chart or a smart chart from Word in Markdown format?
> If not, do you have any suggestions?
> I just want to create a visual way to sum up the main points but I would like to use different shapes, (not just a table of words)
>
> Thanks,
> Kelly
>
> ________________________________________
> From: lets-talk-science <lets-talk-science-bounces@lists.stanford.edu> on behalf of Bruno Beltran <brunobeltran0@gmail.com>
> Sent: Wednesday, January 6, 2016 12:45 PM
> To: lets-talk-science@mailman.stanford.edu
> Subject: [lets-talk-science] Fwd: How to Format Finished Articles for   Publishing
>
> Hi all,
>
> Here's a question that I didn't think to address that will be relevant
> to everyone.
>
> In order to insert an image into the document, simply use the name of
> the image file (including any extensions, like .png, .jpg) as the
> "link name". Since when they are published on the website, the images
> will be in the same folder as the markdown, they will be correctly
> included that way. You won't be able to do this when previewing the
> document on the website I sent out, of course, but a good workaround
> to make sure that it looks the way you want is to upload the image
> somewhere like Dropbox or Google Drive while editing the final version
> of the article and use a link to the uploaded version until you're
> satisfied it looks correct, then simply change those links back to
> just the file name before zipping it up and sending it over.
>
> Also, if you use any unorthodox methods to generate images, please
> make sure that the final image is in a web-friendly format. If you
> don't know that this means, just convert it to a PNG or JPEG file
> before using it, to prevent it from messing up how the article looks
> in different browsers, on phones, etc.
>
> Let me know if any of that is not clear.
>
> Best,
>
> Bruno
>
>
> ---------- Forwarded message ----------
> From: Kelly Jane McGill <kmcgill@stanford.edu>
> Date: Wed, Jan 6, 2016 at 12:14 PM
> Subject: Re: [lets-talk-science] How to Format Finished Articles for Publishing
> To: Bruno Beltran <brunobeltran0@gmail.com>
>
>
> How do you add your own charts or images (power point figures) to
> markdown? Plain text gets rid of the image and if there is no link,
> how can you add it to a markdown document?
>
> Thanks,
> Kelly
>
> ________________________________________
> From: lets-talk-science <lets-talk-science-bounces@lists.stanford.edu>
> on behalf of Bruno Beltran <brunobeltran0@gmail.com>
> Sent: Tuesday, January 5, 2016 8:26 PM
> To: lets-talk-science@mailman.stanford.edu
> Subject: [lets-talk-science] How to Format Finished Articles for Publishing
>
> Hey again guys!
>
> Short version: Please submit each article as a zipped folder
> containing both all the media (e.g. pictures/figures) and a single
> markdown file with the article's contents. You can learn how to use
> markdown in approximately 143 seconds on
> https://github.com/adam-p/markdown-here/wiki/Markdown-Cheatsheet and
> test that your file looks good on https://stackedit.io/editor.
>
> Long version:
>
> A quick update on the file format we're going to want the articles in,
> since some of you have started writing.
>
> Each group is free to (and should) use whatever format (e.g. Word,
> plain text, LaTeX+git) makes it easiest for your groups to get the
> articles written and reviewed. When it comes time for you to submit,
> however, the process of making it look like you want will require that
> you use a web-friendly format. In order to minimize the amount of work
> this will take, we will use "Markdown" syntax to specify the article's
> formatting. Don't be afraid, this does not involve programming,
> learning a new language, or knowing how to cartwheel. The syntax will
> be familiar to anyone that has posted in a forum, reddit, Github, or
> other website that allows formatting comments/posts.
>
> The best way to explain how it works is to simply point you to a cheat
> sheet that shows you how to do everything you could possibly want,
> from italics and bolding to tables, links and images:
>
> https://github.com/adam-p/markdown-here/wiki/Markdown-Cheatsheet
>
> and a site that lets you type in markdown and shows you what it would
> look like on the web, with some built in examples:
>
> https://stackedit.io/editor
>
> In practice, the only thing to keep in mind is that you'll need to put
> two newlines (i.e. hit enter twice) every time you want to start a new
> paragraph.
>
> Please, when submitting an article, put a single markdown or plain
> text file (preferably with file ending ".md") in a folder with all the
> pictures/videos/media that are going to be included in the article,
> zip the folder, and send the zip file to me.
>
> Let me know if there are any questions, and I am happy to help both
> with any technical difficulties that will inevitably arise and with
> getting any particularly fancy formatting to work.
\end{verbatim}

\section{Copyright Law}

This is fairly easy to summarize. If you didn't make it and it's not explicitly
marked as Creative Commons (for reuse) or public domain from a credible source,
then you have to ask the person that owns it for permission.

In particular, this means that if you want to reproduce a figure from a journal,
you \textbf{must} ask the journal's permission since usually the journal is the
copyright holder and not the author.

%%% End document
\end{document}
